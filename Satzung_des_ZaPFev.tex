\documentclass[ngerman]{article}
\usepackage[T1]{fontenc}
\usepackage[utf8]{inputenc}
\setlength{\parskip}{\medskipamount}
\setlength{\parindent}{0pt}
\usepackage{textcomp}

\makeatletter
%%%%%%%%%%%%%%%%%%%%%%%%%%%%%% User specified LaTeX commands.
\usepackage{a4wide}

\makeatother

\usepackage[ngerman]{babel}

% spezielle Silbentrennung:
\hyphenation{Staats-ver-trag Stau-be-cken } 

\begin{document}

\title{SATZUNG}
\date{Stand: \today}

\author{des \textquotedbl{}ZaPF
e.V.\textquotedbl{}}

\maketitle

%\part*{I. Allgemeine Bestimmungen}


\section*{§1 Name, Sitz und Geschäftsjahr}
\begin{enumerate}
\item Der Verein führt den Namen \glqq Zusammenkunft aller Physik Fachschaften\grqq, abgekürzt \glqq ZaPF\grqq. Er strebt die Eintragung in das Vereinsregister und die Anerkennung als gemeinnützig an. Nach der Eintragung führt er den Zusatz \glqq e.V.\grqq.
\item Der Verein hat seinen Sitz in Frankfurt am Main.
\item Das Geschäftsjahr des Vereins läuft vom 1.~Oktober eines Jahres bis zum 30.~September des folgenden Jahres.
\end{enumerate}


\section*{§2 Zweck des Vereins}
Der Verein dient der Förderung der Durchführung der halbjährlichen Treffen der Vertreter der Physikinteressierten der Hochschulen im deutschsprachigen Raum.
Der Verein bezweckt mit diesen Treffen einen Informationsaustausch der einzelnen Studierendenvertretungen und soll so eine überregionale Koordination ermöglichen. Durch Fachvorträge soll über neueste Entwicklungen aus Wissenschaft, Forschung und Technologie berichtet werden.


\section*{§3 Gemeinnützigkeit}
Der Verein verfolgt ausschließlich und unmittelbar gemeinnützige Zwecke im Sinne des Abschnitts \glqq Steuerbegünstigte Zwecke\grqq\ der Abgabenverordnung. Der Verein ist selbstlos tätig, er verfolgt keine eigenwirtschftlichen Zwecke. Mittel des Vereins dürfen nur für satzungsmäßige Zwecke verwendet werden. Die Mitglieder erhalten in
ihrer Eigenschaft als Mitglied keine Zuwendung aus den Mitteln des Vereins. Es darf keine Person durch Ausgaben, die dem Vereinszweck fremd sind oder durch unverhältnismäßig hohe Vergütungen begünstigt werden.


\section*{§4 Erwerb der Mitgliedschaft}
Mitglied des Vereins kann jede natürliche Person werden, die die Satzung des Vereins anerkennt und in der Fachrichtung Physik an einer Hochschule ordentlich immatrikuliert ist.\\
Über Ausnahmen entscheidet der Vorstand.\\
Die Mitgliedschaft wird durch schriftlichen Antrag erworben.


\section*{§5 Beendigung der Mitgliedschaft}
\begin{enumerate}
 \item Die Mitgliedschaft endet durch schriftliche Erklärung des Mitglieds.
 \item Die Mitgliedschaft endet durch Streichung aus der Liste der Mitglieder, falls ein Mitglied über einen Zeitrum von 15 Monaten an keiner Mitgliederversammlung teilgenommen hat.
 \item Die Mitgliederversammlung kann in wichtigen Gründen mit \textthreequarters\ Mehrheit ein Mitglied ausschließen. Als wichtiger Grund zählt ein Verstoß gegen die Vereinsinteressen.
\end{enumerate}


\section*{§6 Finanzen}
Es wird kein Mitgliedsbeitrag erhoben. Der Verein finanziert sich durch öffentliche Mittel, Spenden und Gebühren. Zuwendungen dürfen nicht angenommen werden, wenn sie zu Bedingungen verpflichten, die dem Vereinszweck widersprechen.


\section*{§7 Organe der ZaPF}
Organe der ZaPF sind
\begin{enumerate}
 \item Der Vorstand
 \item Die Mitgliederversammlung
\end{enumerate}


\section*{§8 Die Mitgliederversammlung}
\begin{enumerate}
 \item Die Mitgliederversammlung ist das höchste Organ des Vereins und findet mindestens einmal jährlich statt.
 \item Die Mitgliederversammlung ist bei ordnungsgemäßer Einberufung immer beschlussfähig.
 \item Jedes Mitglied ist stimmberechtigt und hat genau eine Stimme.
 \item Es wird ein Protokoll geführt.
 \item Die Einladung zur Mitgliederversammlung erfolgt auf dem halbjährlichen Treffen der Vertreter der Physikstudierenden im jeweils vorangehenden Semester unter Angabe einer Tagesordnung. Zusätzlich ergeht binnen drei Wochen vor der Mitgliederversammlung eine Einladung an die Mitglieder in Textform unter Angabe der Tagesordnung.
 \item Die Tagesordnung kann zu Beginn der Mitgliederversammlung geändert werden. Hiervon ist der Tagesordnungspunkt `Satzungsänderungen' ausgeschlossen.
 \item Sie wählt den Vorstand, nimmt dessen Rechenschaftsbericht entgegen, setzt Kassenprüfer ein und legt den Haushaltsplan des Vereins fest.
\end{enumerate}


\section*{§9 Vorstand}
\begin{enumerate}
 \item Der Vorstand besteht aus bis zu sieben Personen.
 \item Jedes Vorstandsmitglied ist einzeln vertretungsberechtigt.
 \item Der Vorstand wird durch die Mitgliederversammlung mit 2/3 der Stimmen der anwesenden Mitglieder des Vereins gewählt. Kommt diese Mehrheit in zwei Wahlgängen nicht zustande, so entscheidet im dritten Wahlgang die einfache Mehrheit.
 \item Der Vorstand beruft die Mitgliederversammlung ein, erarbeitet den Haushaltsplan und erstellt Berichte für die Mitgliederversammlung, insbesondere den Rechenschaftsbericht.
 \item Der Vorstand wählt aus seiner Mitte einen Kassenwart.
\end{enumerate}


\section*{§10 Entlastung des Vorstandes}
Der Vorstand legt am Ende seiner Amtszeit der Mitgliederversammlung einen Rechenschaftsbericht vor. Insbesondere muss der entsprechende Finanzbericht durch die Finanzprüfer bestätigt werden.\\
Der Vorstand kann für nicht abgeschlossene Rechtsgeschäfte, die aus seiner Tätigkeit in der Amtszeit herrühren, nicht entlastet werden. Die Entlastung ist dann auf der nächsten Mitgliederversammlung nach Abschluss dieser Rechtsgeschäfte zu beantragen.


\section*{§11 Satzungsänderung}
Eine Satzungsänderung kann nur beschlossen werden, wenn sie vorher auf der Tagesordnung angekündigt war.\\
Eine Satzungsänderung kann nur durch die Mitgliederversammlung mit 2/3 der Stimmen der anwesenden Mitglieder des Vereins erfolgen.\\
Anträge auf Aufnahme eines Tagesordnungspunktes \glqq Satzungsänderung\grqq\ müssen mindestens vier Wochen vor der Mitgliederversammlung unter ausführlicher Angabe von Gründen und konkreten Entwürfen der Änderungen beim Vorstand eingereicht werden.


\section*{§12 Auflösung}
\begin{enumerate}
 \item Die Auflösung der ZaPF kann von der Mitgliederversammlung nur mit 3/4 Mehrheit der Stimmen der anwesenden Mitglieder des Vereins erfolgen.
 \item Bei Auflösung des Vereins oder Wegfall der steuerbegünstigten Zwecke fällt das Vereinsvermögen an eine von der Mitgliederversammlung zu bestimmende gemeinnützige Organisation. Beschlüsse über die künftige Verwendung des Vermögens dürfen erst nach Einwilligung des Finanzamts durchgeführt werden.
 \item Bei der Auflösung wählt die Mitgliederversammlung zwei Mitglieder zu vertretungsberechtigten Liquidatoren.
\end{enumerate}


\end{document}
