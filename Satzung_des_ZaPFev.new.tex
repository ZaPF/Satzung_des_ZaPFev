\documentclass[ngerman]{article}
\usepackage[T1]{fontenc}
\usepackage[utf8]{inputenc}
\setlength{\parskip}{\medskipamount}
\setlength{\parindent}{0pt}
\usepackage{textcomp}
\usepackage{nicefrac}

\makeatletter
%%%%%%%%%%%%%%%%%%%%%%%%%%%%%% User specified LaTeX commands.
\usepackage{a4wide}

\makeatother

\usepackage[ngerman]{babel}

% spezielle Silbentrennung:
\hyphenation{Staats-ver-trag Stau-be-cken}

\begin{document}

\title{SATZUNG}
\date{Stand: Beschlossene Änderungen vom 30.~Mai~2015}

\author{des \textquotedbl{}ZaPF
e.V.\textquotedbl{}}

\maketitle

%\part*{I. Allgemeine Bestimmungen}


\section*{§1 Name, Sitz und Geschäftsjahr}
\begin{enumerate}
\item Der Verein führt den Namen \glqq Zusammenkunft aller Physik Fachschaften e.V.\grqq.
\item Der Verein hat seinen Sitz in Frankfurt am Main.
\item Das Geschäftsjahr ist das Kalenderjahr.
\end{enumerate}


\section*{§2 Zweck des Vereins}
Der Verein dient der Förderung der Durchführung der halbjährlichen Treffen der Vertreter/innen der Physikstudierenden der Hochschulen im deutschsprachigen Raum.
Der Verein bezweckt mit diesen Treffen einen Informationsaustausch der einzelnen Studierendenvertretungen und soll so eine überregionale Koordination ermöglichen. Durch Fachvorträge soll über neueste Entwicklungen aus Wissenschaft, Forschung und Technologie berichtet werden.


\section*{§3 Gemeinnützigkeit}
Der Verein verfolgt ausschließlich und unmittelbar gemeinnützige Zwecke im Sinne des Abschnitts \glqq Steuerbegünstigte Zwecke\grqq\ der Abgabenverordnung. Der Verein ist selbstlos tätig, er verfolgt keine eigenwirtschftlichen Zwecke. Mittel des Vereins dürfen nur für satzungsmäßige Zwecke verwendet werden. Die Mitglieder erhalten in
ihrer Eigenschaft als Mitglied keine Zuwendung aus den Mitteln des Vereins. Es darf keine Person durch Ausgaben, die dem Vereinszweck fremd sind oder durch unverhältnismäßig hohe Vergütungen begünstigt werden.


\section*{§4 Erwerb der Mitgliedschaft}
Mitglied des Vereins kann jede natürliche Person werden, die die Satzung des Vereins anerkennt.\\

Außerordentliches Mitglied (\glqq Fördermitglied\grqq) kann jede natürliche und juristische Person werden, die die Satzung des Vereins anerkennt.
Über Ausnahmen entscheidet der Vorstand.\\
Die Mitgliedschaft wird durch schriftlichen Antrag erworben.


\section*{§5 Beendigung der Mitgliedschaft}
\begin{enumerate}
 \item Die Mitgliedschaft ordentlicher Mitglieder endet unverzüglich durch
  \begin{enumerate}
      \item schriftliche Erklärung des Mitglieds,
      \item Streichung aus der Liste der Mitglieder, wenn über einen Zeitraum von 15 Monaten an keiner Mitgliederversammlung teilgenommen wurde.
  \end{enumerate}
  \item Die Mitgliedschaft außerordentlicher Mitglieder endet zum Ende des Kalenderjahrs durch
  \begin{enumerate}
    \item schriftliche Erklärung des Mitglieds,
    \item Nichtzahlung des Mitgliedsbeitrags bis zum 31.\, Dezember eines Jahres.
  \end{enumerate}
 \item Die Mitgliederversammlung kann in wichtigen Gründen mit \nicefrac{3}{4}\ Mehrheit ein Mitglied ausschließen. Als wichtiger Grund zählt ein Verstoß gegen die Vereinsinteressen.
\end{enumerate}


\section*{§6 Finanzen}
Es wird kein Mitgliedsbeitrag für ordentliche Mitglieder erhoben. Der Verein finanziert sich durch die Mitgliedsbeiträge der außerordentlichen Mitglieder, öffentliche Mittel, Spenden und Gebühren. Zuwendungen dürfen nicht angenommen werden, wenn sie zu Bedingungen verpflichten, die dem Vereinszweck widersprechen.
Näheres regelt die Beitragsordnung.


\section*{§7 Organe des Vereins}
Organe des Vereins sind
\begin{enumerate}
 \item Der Vorstand
 \item Die Mitgliederversammlung
\end{enumerate}

\section*{§8 Die Mitgliederversammlung}
\begin{enumerate}
 \item Die Mitgliederversammlung ist das höchste Organ des Vereins und findet mindestens einmal jährlich statt.
 \item Die Mitgliederversammlung ist bei ordnungsgemäßer Einberufung beschlussfähig.
 \item Jedes ordentliche Mitglied ist stimmberechtigt und hat genau eine Stimme. Außerordentliche Mitglieder nehmen beratend teil.
 \item Es wird ein Protokoll geführt.
 \item Die Einladung zur Mitgliederversammlung erfolgt durch den Vorstand mindestens drei Wochen im Voraus in Textform an die Mitglieder unter Angabe der Tagesordnung.
 \item Die Tagesordnung kann zu Beginn der Mitgliederversammlung geändert werden. Hiervon ist der Tagesordnungspunkt `Satzungsänderungen' ausgeschlossen.
 \item Der Mitgliederversammlung kommen insbesondere folgende Aufgaben zu:
 \begin{itemize}
  \item Wahl des Vorstands (mindestens einmal jährlich)
  \item Bestellung einer/eines Kassenprüferin/Kassenprüfers
  \item Entgegennahme der Tätigkeits- und Rechenschaftsberichte
  \item Entgegennahme des Finanzberichts
  \item Entlastung des Vorstands
  \item Festlegung der Beitragsordnung
  \item Satzungsänderungen (\nicefrac{2}{3}\ Mehrheit)
  \item Auflösung des Vereins (\nicefrac{3}{4}\ Mehrheit)
 \end{itemize}
\end{enumerate}


\section*{§9 Vorstand}
\begin{enumerate}
 \item Der Vorstand nach § 26 BGB besteht aus der/dem Vorsitzenden, der/dem Kassenführer/in und bis zu sechs weiteren Personen, die mit ihrer Wahl mit besonderen Aufgaben betraut werden können.
 \item Jedes Vorstandsmitglied ist einzeln vertretungsberechtigt.
 \item Der Vorstand wird durch die Mitgliederversammlung mit \nicefrac{2}{3}\ der Stimmen der anwesenden stimmberechtigten Mitglieder des Vereins gewählt.
       Kommt diese Mehrheit in zwei Wahlgängen nicht zustande, so entscheidet im dritten Wahlgang die einfache Mehrheit.
       Frühere Vorstandsmitglieder, die nicht erneut gewählt wurden, scheiden automatisch aus dem Vorstand aus.
 \item Die Aufgaben des Vorsitzenden des Vorstands sind insbesondere
  \begin{enumerate}
   \item die Führung des Vereins im strategischen und grundsätzlichen Bereich,
   \item die Berufung der Mitgliederversammlung,
   \item die Verwaltung der Mitglieder,
   \item die Unterzeichnung der Mitgliederversammlungsprotokolle,
   \item die Einreichung der Mitgliederversammlungsprotokolle beim Amtsgericht.
  \end{enumerate}
  Einzelne dieser Aufgaben können stattdessen auch anderen Vorstandsmitgliedern bei ihrer Wahl übertragen werden.

 \item Die Aufgaben der/des Kassenführerin/Kassenführers sind insbesondere:
  \begin{enumerate}
   \item die Führung der Bücher des Vereins,
   \item die Erstellung des Finanzberichts,
   \item die Einreichung der Steuererklärung beim Finanzamt.
  \end{enumerate}

 \item Zur Aufgabe eines jeden Vorstandsmitglieds gehört die Erstellung eines schriftlichen Tätigkeits- und Rechenschaftsberichts.

  \item Als Aufgabengebiete, zu denen Vorstände gewählt werden können kommen insbesondere in Betracht:
  \begin{enumerate}
   \item Vertretung des Vereins bezüglich der Durchführung einer bestimmten Zusammenkunft aller Physik-Fachschaften vor Ort,
   \item Akquise und Betreuung von Fördermitgliedern und Spendern.
  \end{enumerate}
\end{enumerate}


\section*{§10 Entlastung des Vorstandes}
Die Vorstandsmitglieder legen am Ende ihrer Amtszeit der Mitgliederversammlung den Tätigkeits- und Rechenschaftsbericht vor.
Insbesondere müssen diese, sowie der Finanzbericht des Kassenführers, durch die/den Kassenprüfer/in bestätigt werden.
Die Entlastung erfolgt auf Grundlage der Berichte.\\
Ein Vorstandsmitglied kann für nicht abgeschlossene Rechtsgeschäfte, die aus dessen Tätigkeit in der Amtszeit herrühren, nicht entlastet werden. Die Entlastung ist dann auf der nächsten Mitgliederversammlung nach Abschluss dieser Rechtsgeschäfte zu beantragen.\\
Ein Vorstandsmitglied kann nicht über seine eigene Entlastung abstimmen.

\section*{§11 Satzungsänderung}
Eine Satzungsänderung kann nur beschlossen werden, wenn sie vorher auf der Tagesordnung angekündigt war.\\
Eine Satzungsänderung kann nur durch die Mitgliederversammlung mit \nicefrac{2}{3}\ der Stimmen der anwesenden stimmberechtigten Mitglieder des Vereins erfolgen.\\
Anträge auf Aufnahme eines Tagesordnungspunktes \glqq Satzungsänderung\grqq\ müssen mindestens vier Wochen vor der Mitgliederversammlung unter ausführlicher Angabe von Gründen und konkreten Entwürfen der Änderungen beim Vorstand eingereicht werden.


\section*{§12 Auflösung}
\begin{enumerate}
 \item Die Auflösung des Vereins kann von der Mitgliederversammlung nur mit \nicefrac{3}{4}\ Mehrheit der Stimmen der anwesenden stimmberechtigten Mitglieder des Vereins erfolgen.
 \item Bei Auflösung oder Aufhebung der Körperschaft oder bei Wegfall steuerbegünstigter Zwecke fällt das Vermögen der Körperschaft an eine juristische Person des öffentlichen Rechts oder eine andere steuerbegünstigte Körperschaft zwecks Verwendung für die Förderung von Erziehung, Volks- und Berufsbildung einschließlich der Studentenhilfe.
 \item Bei der Auflösung wählt die Mitgliederversammlung zwei Mitglieder zu vertretungsberechtigten Liquidatoren.
\end{enumerate}

\section*{§13 Salvatorische Klausel}
\begin{enumerate}
  \item Sollte eine der Bestimmungen dieser Satzung ganz oder teilweise rechtswidrig oder unwirksam sein oder werden, so wird die Gültigkeit der übrigen Bestimmungen dadurch nicht berührt. In einem solchen Fall ist die Satzung vielmehr ihrem Sinne gemäß zur Durchführung zu bringen. Beruht die Ungültigkeit auf einer Leistungs- oder Zeitbestimmung, so tritt an ihrer Stelle das gesetzlich zulässige Maß.

  \item Die rechtswidrige oder unwirksame Bestimmung ist unverzüglich durch Beschluss der nächsten Mitgliederversammlung zu ersetzen.
\end{enumerate}

\section*{§14 Schlussbestimmungen}
Diese Satzung wurde in der Mitgliederversammlung am 30.~Mai~2015 beschlossen. Sie tritt mit der Eintragung des Vereins in das Vereinsregister in Kraft. Alle vorherigen Satzungen treten damit außer Kraft.

Aachen, den 30.~Mai~2015

\end{document}
